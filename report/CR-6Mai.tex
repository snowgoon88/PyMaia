\documentclass[12pt]{article}

\usepackage[french]{babel}
\usepackage[utf8]{inputenc}
\usepackage[OT1]{fontenc}
\usepackage{graphicx}
\usepackage{verbatim}
\usepackage{amsfonts}


\title{Compte Rendu du 06/05}
\author{}
\date{}

\begin{document}
\maketitle

\paragraph{Ordre du jour}
Fixer les premières pistes de réflexion, ainsi que nos objectifs quand à l'étude des ESN.

\section{Les tâches}
\paragraph{MackeyGlass}
Permets de tester l'aptitude de l'ESN à prédire/générer des fonctions chaotiques, et dans une moindre mesure de tester la mémoire du réseau.
\paragraph{Séquences}
Permets essentiellement de tester la faculté de mémorisation du réseau, mais aussi son comportement face à l'aléatoire.
\paragraph{Trajectoires}
Permets de tester la performance du réseau à classifier un ensemble déterministe de trajectoires.

\section{Les paramètres}
Nous aimerions mieux appréhender l'impacte de certains paramètres sur les performances de l'ESN, en particulier :
\begin{description}
\item[$\alpha$] le taux de fuite
\item[$\rho$] le rayon spectral
\item[$N$] la taille du réservoir
\item[$W_{in}$] la matrice d'entrée, en particulier son échelle
\item[$W$] la matrice du réservoir, en particulier sa \textit{sparsity}
\item[$\tau$] le temps de relaxation
\end{description}

\section{Pistes de réflexions \& Intuitions}

\paragraph{Batch vs Online}
Est il possible qu'un apprentissage en ligne avec N1 échantillons soit équivalent à un apprentissage en lots avec N2 échantillons ? Que ce passe t'il si N1 = N2 ? L'apprentissage en ligne va nécessiter le développement d'un nouvel algorithme afin d'apprendre les poids de $W_{out}$, dans un premier temps nous utiliserons un perceptron multi-couche pour faire cet apprentissage.

\begin{itemize}
\item Batch devrait être plus performant que Online
\end{itemize}

\paragraph{Nombre d'échantillons}
Quel est l'influence du nombre d'échantillons dans le corpus d'apprentissage ? En quoi la taille du corpus influence t'elle la faculté de généralisation du réseau ? Y a t'il risque de sur-apprentissage ? Est il possible de déterminer une taille optimale de corpus (en fonction des paramètres du réservoir) pour éviter ce phénomène ?

\begin{itemize}
\item 
\end{itemize}

\paragraph{Capacité}
Nous nous intéressons ici à la capacité du réservoir, c'est à dire nombre de classe que celui-ci peut différencier, mais aussi à la capacitité de mémorisation, c'est à dire le nombre d'échantillon mémorisé par le réservoir. Dans le cas d'un $y(t)$ dépendant de $u(t-\Delta t)$, quel peut être le $\Delta t$ maximum supporté par l'ESN ?

\begin{itemize}
\item grandement lié à $\alpha$
\item certainement lié à $N$
\end{itemize}


\paragraph{Généralisation}
L'ESN est il capable de généraliser ou a t'il une tendance naturelle au surapprentissage ? Dans le cas d'une classification, quel est le bassin d'attraction des différents classes ?

\begin{itemize}
\item certainement lié à $N$
\end{itemize}


\paragraph{Robustesse}
Nous nous intéressons ici à la robustesse d'un ESN face au bruit. Comment se comporte il après un apprentissage avec des enchantillons bruités ? Comment l'ESN se comporte il si le corpus de test est bruité ? Et si les deux le sont ?

\begin{itemize}
\item Fortement lié avec la généralisation, si il y a une bonne généralisation alors le bruit ne devrait pas trop déranger durant la phase d'apprentissage.
\end{itemize}


\paragraph{Précision}
Quelle est la précision de l'ESN ? Quelle est la moyenne des erreurs ainsi que l'écart type ?

\begin{itemize}
\item
\end{itemize}

\paragraph{Rappel}
L'ESN à t'il correctement appris le corpus d'apprentissage ?

\begin{itemize}
\item De part la régression linéaire, les résultats en rappel devrait toujours être relativement bon.
\end{itemize}


\section{Méthodologie}

SOON =D

\end{document}