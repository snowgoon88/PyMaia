\documentclass[12pt]{article}

\usepackage[french]{babel}
\usepackage[utf8]{inputenc}
\usepackage[OT1]{fontenc}
\usepackage{graphicx}
\usepackage{verbatim}
\usepackage{amsfonts}


\title{Compte Rendu de Réunion}
\date{6 Mai 2014}
\author{}

\begin{document}
\maketitle

\paragraph{Ordre du jour}
Fixer les premières pistes de réflexion, ainsi que nos objectifs quant à l'étude des ESN.

\section{Les tâches}
\paragraph{MackeyGlass}
Permet de tester l'aptitude de l'ESN à prédire/générer des fonctions chaotiques, et dans une moindre mesure de tester la mémoire du réseau.
\paragraph{Séquences}
Permet essentiellement de tester la faculté de mémorisation du réseau, mais aussi son comportement face à l'aléatoire.
\paragraph{Trajectoires}
Permet de tester la performance du réseau à classifier un ensemble fini de trajectoires déterministes.

\section{Les paramètres}
Nous aimerions mieux appréhender l'impact de certains paramètres sur les performances de l'ESN, en particulier :
\begin{description}
\item[$\alpha$] le taux de fuite
\item[$\rho$] le rayon spectral de $W$
\item[$N$] la taille du réservoir
\item[$W_{in}$] la matrice d'entrée
\item[$W$] la matrice du réservoir, en particulier sa \textit{sparsity}
\item[$\tau$] le temps de relaxation
\end{description}

\section{Pistes de réflexions \& Intuitions}

\paragraph{Batch vs Online}
Est il possible qu'un apprentissage en ligne avec N1 échantillons soit équivalent à un apprentissage en lots avec N2 échantillons ? Que se passe t'il si N1 = N2 ? L'apprentissage en ligne va nécessiter soit le développement d'un nouvel algorithme afin d'apprendre les poids de $W_{out}$, par exemple en utilisant un perceptron pour faire cet apprentissage, soit l'utilisation de la méthode d'apprentissage en ligne \textit{Backpropagation-Decorrelation} (BPDC) proposé par \cite{Steil05}.

\begin{itemize}
\item Batch devrait être plus performant que Online, car il minimise directement l'erreur sur l'ensemble du corpus d'apprentissage, évitant ainsi les possibilités de tomber dans un minimum local comme dans un apprentissage avec une rétropropagation de l'erreur classique.
\end{itemize}

\paragraph{Nombre d'échantillons}
Quel est l'influence du nombre d'échantillons dans le corpus d'apprentissage ? En quoi la taille du corpus influence t'elle la faculté de généralisation du réseau ? Y a t'il risque de sur-apprentissage ? Est-il possible de déterminer une taille optimale de corpus (en fonction des paramètres du réservoir) pour éviter ce phénomène ?

\begin{itemize}
\item 
\end{itemize}

\paragraph{Capacité \& Mémoire}
Nous nous intéressons ici à la capacité du réservoir, c'est à dire nombre de classe que celui-ci peut différencier, mais aussi à sa mémoire, c'est à dire le nombre d'entrées artificiellement mémorisées par le réservoir grâce aux connexions récurrentes. Dans le cas d'un $y(t)$ dépendant de $u(t-\Delta t)$, quel peut être le $\Delta t$ maximum supporté par l'ESN ?

\begin{itemize}
\item grandement lié à $\alpha$ et $\rho$
\item certainement lié à $N$
\end{itemize}


\paragraph{Généralisation}
L'ESN est-il capable de généraliser ou a-t-il une tendance naturelle au sur-apprentissage ? Dans le cas d'une classification, quel est le bassin d'attraction des différentes classes ?

\begin{itemize}
\item certainement lié à $N$
\end{itemize}


\paragraph{Robustesse}
Nous nous intéressons ici à la robustesse d'un ESN face au bruit. Comment se comporte il après un apprentissage avec des enchantillons bruités ? Comment l'ESN se comporte il si le corpus de test est bruité ? Et si les deux le sont ?

\begin{itemize}
\item Fortement lié avec la généralisation, si il y a une bonne généralisation alors le bruit ne devrait pas trop déranger durant la phase d'apprentissage.
\item Bruiter les échantillons du corpus d'apprentissage pourrait éventuellement aider à mieux généraliser.
\end{itemize}


\paragraph{Précision}
Quelle est la précision de l'ESN ? Quelle est la moyenne des erreurs ainsi que l'écart type ?

\begin{itemize}
\item
\end{itemize}

\paragraph{Rappel}
L'ESN à t'il correctement appris le corpus d'apprentissage ?

\begin{itemize}
\item De part la régression linéaire, les résultats en rappel devrait être relativement bon, sous la condition que le réservoir fournisse \"assez\" d'états internes décorrelés pour permettre de reproduire la dynamique du signal d'entrée.
\end{itemize}

\paragraph{Feedback}
En quoi influent les synapses de feedback pour un RNN ?

\begin{itemize}
\item
\end{itemize}


\section{Méthodologie}

\subsection{} 

\bibliographystyle{apalike}
\bibliography{reservoir}

\end{document}